
% Misc. packages
\usepackage[T1]{fontenc}
\usepackage{url}
\usepackage{amsfonts}
\usepackage[absolute]{textpos}
\usepackage{amssymb}
\usepackage{amsmath} 
\usepackage{amsthm}

% Graphics stuff
\usepackage[usenames,dvipsnames]{color}
\usepackage{graphicx}


% My favourite macros
\newcommand{\N}{\mathbb{N}} %natural numbers
\newcommand{\Z}{\mathbb{Z}} %integers
\newcommand{\Q}{\mathbb{Q}} %rationals
\newcommand{\R}{\mathbb{R}} %reals

\newcommand{\G}{\mathcal{G}} % fancy G
\newcommand{\A}{\mathcal{A}} % fancy A
\newcommand{\bO}{\mathcal{O}} % fancy O

\newcommand{\eos}{\#}
\newcommand{\rank}{\textsc{rank}}


% enumitem for controlling enumerate and itemize environments; usefull for saving space
\usepackage{enumitem}


% Fonts
%
% Palantino - Helvetica - Courier.
% I haven't really spent time to figure out how to best match the official university style with latex font packages, as I use XeTex myself...
\usepackage{mathpazo}
\linespread{1.10}
\usepackage[scaled]{helvet}
\usepackage{courier}

% Colours
\definecolor{sciorange}{RGB}{252,163,17}
\definecolor{unigray}{RGB}{140,140,140}
\definecolor{first}{RGB}{180,30,10}
\definecolor{second}{RGB}{10,120,130}
\definecolor{new}{RGB}{190,50,10}
\definecolor{wavelet}{RGB}{70,140,220}
%\definecolor{improved}{RGB}{80,200,80}
\definecolor{improved}{RGB}{10,100,160}
\definecolor{prior}{RGB}{140,140,140}

% Textpos to manually position blocks of text on the page
\usepackage[absolute]{textpos}

% We define 1 mm grid for positioning the text blocks on the page
% The idea is to leave 10 mm marginals to all sides; the three text columns are 60 mm wide with 5 mm space between columns.
\setlength{\TPHorizModule}{1mm}
\setlength{\TPVertModule}{1mm}

% The origin is set to right below the main title; this means that main title blocks have negative y-coordinate. There is actually no good reason for this, I just happened to do this that way.
\textblockorigin{10mm}{48.5mm}


% parindent is set to zero, because it looks better in posters
% you could also add some space between paragraphs here, but I use manual vertical spaces in this sample
\setlength{\parindent}{0pt}

%Nice table cell formatting for multirow cells:
\newcommand{\specialcell}[2][c]{%
  \begin{tabular}[#1]{@{}c@{}}#2\end{tabular}}
\usepackage{multicol}
 
