 {\sffamily\normalsize{\color{sciorange}RESULTS}}\vspace{-7mm}\small\\
 \footnotesize 
\begin{multicols}{3}
The above results were achieved with the PyPy just-in-time compiler for
Python on a quad-core 64-bit Intel Core i5 machine with 6 megabytes of
cache and 8 gigabytes of RAM.  The results should be taken with a grain
of salt due to our use of a high-level language with a JIT compiler:
non-transparent optimizations may have been performed at any step of the
execution process.  For a direct comparison, a lower-level language should
be used.\\

Our choice of a fixed alphabet of $256$ symbols for MSD radix sort definitely
hurt its performance: less than half of the buckets allocated at each
partitioning step are ever used to hold any strings with every other dataset
besides WORDS.\\

While fast in theory, quicksort versions that were not sorting in-place were
rather sluggish.  Some of the peformance degradation could have been caused
by $O(n \log n)$  memory requirement.  Coupled with read/write access to
non-contiguous memory areas the resulting cache-misses caused performance
penalties.\\

It was interesting to note how much better the in-place algorithms (burstsort
and in-place multikey quicksort) performed on the more demanding datasets.
Despite being a high-level language, it appears there are still performance
gains to be had in programming close to the hardware in Python.
\end{multicols}
