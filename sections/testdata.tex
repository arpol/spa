% Three-column stuff
%
% The vertical size of the columns depends on the content, so unfortunately you have to manually move contents around

\begin{textblock}{180}(0,160)
\sffamily\normalsize{\color{sciorange}DATASETS}\small\\
\rule[3mm]{190mm}{0.1pt}
\end{textblock} 
\begin{textblock}{190}(0,163)
 \footnotesize 
%TODO: Create bibliography for this
\begin{multicols}{3}
With the exception of the URLs dataset, all datasets were retrieved from
the Pizza \& Chili Corpus [2]; the URL dataset is the one used by Ranjan Sinha
in his original burstsort paper [3]. The algorithms were tested on a sample
of 100 and 200 megabytes with each dataset.\vspace{1.5mm}\\
{\sffamily\normalsize{\color{sciorange}DNA}}\small\\
\footnotesize 
The DNA dataset consists of sequences of nucleotide codes, all exactly $3732300$
characters in length.  This is by far the easiest dataset, having the smallest
number of strings and the smallest LCP array sum; very little of the extremely
long strings is actually required for sorting them.\vspace{1.5mm}\\
{\sffamily\normalsize{\color{sciorange}URLS}}\small\\
\footnotesize 
The URLS dataset consists of several web addresses.  Due to most common URLs
having similar prefixes, as well as the dataset containing several duplicate
URLs, this dataset has the highest LCP array sum, though not significantly
higher than the WORDS dataset.\vspace{1.5mm}\\
{\sffamily\normalsize{\color{sciorange}WORDS}}\small\\
\footnotesize 
The WORDS dataset is a modification of the ENGLISH dataset of the Pizza \& Chili
Corpus, constructed by splitting each word on its own line in order to make our
algorithms sort individual words instead of entire lines.  The dataset thus
consists of very many very short strings, with a few outliers due to formatting
markup in the source file.  The dataset also ranks second highest in LCP array
sum size and highest in alphabet size, due to common words appearing hundreds of
times in the text and some loan words using characters not in the English
alphabet.

\end{multicols}

\end{textblock} 

