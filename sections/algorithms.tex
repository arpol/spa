\begin{textblock}{190}(0,31)
\sffamily\normalsize{\color{sciorange}ALGORITHMS}\small\\
\rule[3mm]{190mm}{0.1pt}
\end{textblock}
\begin{textblock}{190}(0,31)
\begin{multicols}{3}
\footnotesize
String processing algorithms distinguish themselves from naive comparison
algorithms by maintaining knowledge of the lengths of the *longest common
prefixes* (LCP) of pairs of input strings as they sort them, which they use to
avoid redundant comparisons.  The *LCP array* of a set of strings, by
extension, is defined as follows:
\end{multicols}
\end{textblock}
%leaving the column location definitions outside of the actual content
\begin{textblock}{60}(0,58)
    {\sffamily\normalsize{\color{sciorange} MSD RADIX SORT}}\vspace{1mm}\\
\footnotesize 
MSD radix sort first partitions the strings into different buckets based on
first symbol is, then recursively partitions *those* buckets based on what
the second symbol is, and so on.  When only single-element buckets or buckets
containing only strings shorter than the recursion depth are left, the results
are concatenated and output.\\

{\em Highlight and underline these to illustrate the partitioning.}
\begin{quote}
    {\color{green}a}ctor\\
    {\color{sciorange}al}locate\\
    {\color{sciorange}al}pha\\
    {\color{red}b}eta\\
    {\color{red}b}yproduct\\
\end{quote}
MSD radix sort never needs to process a symbol twice, technically giving it
$O(L(R) + n)$ complexity assuming a finite alphabet.  However, the complexity
is dominated by the bucket container data structure: if $\sigma$ is the size of the
alphabet and if the buckets are stored in a binary search tree, each addition
takes $O(\log \sigma)$ time.  If they are stored in an array or a hash table, merging
takes $\Theta(\sigma)$ time.
\end{textblock}

\begin{textblock}{60}(65,58)
    {\sffamily\normalsize{\color{sciorange} QUICKSORT ALGORITHMS}}\vspace{1mm}\\
\footnotesize 
Quicksort text  
Quicksort text 
Quicksort text  
Quicksort text 
Quicksort text  
Quicksort text 
Quicksort text  
Quicksort text 
Quicksort text  
Quicksort text 
Quicksort text  
Quicksort text 
Quicksort text  
Quicksort text 
Quicksort text  
Quicksort text 
\end{textblock}

\begin{textblock}{60}(130,58)
    {\sffamily\normalsize{\color{sciorange} BURST SORT}}\vspace{1mm}\\
\footnotesize 
Burst sort text    
Burst sort text    
Burst sort text    

Burst sort text    
Burst sort text    
Burst sort text    
Burst sort text    
Burst sort text    

Burst sort text    
Burst sort text    

Burst sort text    
Burst sort text    

Burst sort text   

\end{textblock}
