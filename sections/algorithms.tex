\begin{textblock}{190}(0,31)
\sffamily\normalsize{\color{sciorange}ALGORITHMS}\small\\
\rule[3mm]{190mm}{0.1pt}
\end{textblock}
\begin{textblock}{190}(0,31)
\begin{multicols}{3}
\footnotesize
String processing algorithms distinguish themselves from naive comparison
algorithms by maintaining knowledge of the lengths of the *longest common
prefixes* (LCP) of pairs of input strings as they sort them, which they use to
avoid redundant comparisons.  The *LCP array* of a set of strings, by
extension, is defined as follows:
\end{multicols}
\end{textblock}
%leaving the column location definitions outside of the actual content
\begin{textblock}{60}(0,58)
    {\sffamily\normalsize{\color{sciorange} MSD RADIX SORT}}\vspace{1mm}\\
\footnotesize 
MSD (most significant digit) radix sort is a divide-and-conquer algorithm that
partitions the strings based on their character at a given position.  The
comparison position starts from 0 and increases with one at every recursion
level.  No position, then, is visited twice; and if the algorithm does not
attempt to partition buckets of size 1 or consisting entirely of strings
shorter than the recursion depth, each string is visited at most one more time
than the length of its shortest distinguishing prefix.  Thus, the partitioning
takes at most $O(L(R)+n)$ time, where $L(R)$ is the sum of the LCP array.

However, efficient implementations require the buckets to be implemented as an
array of linked lists in order to avoid the overhead of binary search tree
insertions and lookups.  This allows true constant time insertion to buckets,
but wastes time and memory if the strings use only a fraction of the alphabet
for which MSD radix sort allocates space.  Likewise, if the number of strings
is smaller than the size of the alphabet, standard comparison based string
sorting algorithms outperform MSD radix sort.

Our implementation uses a fixed alphabet size of 256 and falls back to ternary
quicksort when the size of the bucket drops below it.

\end{textblock}

\begin{textblock}{60}(65,58)
    {\sffamily\normalsize{\color{sciorange} QUICKSORT}}\vspace{1mm}\\
\footnotesize 
String quicksort operates by recursively partioning the list of strings corresponding to their lexicographical order compared to the pivot string. It sorts strings in time $O(L(R) + n \log n)$.\\ 

Good pivot selection reduces the maximum depth of the recursion tree. For quicksort we selected the pivot as the median of the first, the last, and the middle string in the list.\\

Ternary quicksort partitions the strings based on whole string comparison. Resulting sets are:
    
\begin{itemize}
    \item Equal to the pivot
    \item Lexicographically smaller than the pivot
    \item Lexicographically larger than the pivot
\end{itemize}

Multikey quicksort partitions using single character comparison, an extra partition for strings with the currently compared character being their last is created.\\

Both algorithms recursively partition the strings until each partition contains only one string. The strings are returned in the reverse order of recursion resulting in a sorted array of strings. 





\end{textblock}

\begin{textblock}{60}(130,58)
    {\sffamily\normalsize{\color{sciorange} BURST SORT}}\vspace{1mm}\\
\footnotesize 
Burst sort text    
Burst sort text    
Burst sort text    

Burst sort text    
Burst sort text    
Burst sort text    
Burst sort text    
Burst sort text    

Burst sort text    
Burst sort text    

Burst sort text    
Burst sort text    

Burst sort text   

\end{textblock}
