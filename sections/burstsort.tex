{\sffamily\normalsize{\color{sciorange} BURSTSORT}}\vspace{1mm}\\
\footnotesize 
Burstsort partitions the input strings into growable but limited-sized
buckets that are afterwards sorted using an auxiliary sorting
algorithm, such as multikey quicksort. Unlike many distribution sorts,
Burstsort reads the input sequentially in order to avoid cache misses.
The partitioning limits the input length for the auxialiary sort and
can thus improve its cache-efficiency as well.\\

Burstsort uses a special data structure, \emph{burst trie}, to store
the buckets and maintain their order. The burst trie is a special kind
of trie whose leaf nodes are growable buckets containing pointers to
the suffixes of the inserted strings while the path from the root to
the leaf node via single-character edges represents the prefix of the
strings just as with a regular trie.\\

A new operation, \emph{burst}, is needed when a full bucket is already
at the maximum capacity and a new string is being inserted into
it. Burst creates a new bucket for each character in the alphabet and
redistributes the suffixes in the bursting bucket into them.\\

