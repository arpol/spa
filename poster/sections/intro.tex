% Three-column stuff
%
% The vertical size of the columns depends on the content, so unfortunately you have to manually move contents around

% First column
\begin{textblock}{60}(0,20)
  {\sffamily\normalsize{\color{sciorange}BURROWS--WHEELER
      TRANSFORM}}\vspace{1mm}\\ % Titles among the main text are made like this, not by using \section
  \footnotesize 
The Burrows--Wheeler transform (BWT) is an invertible text transform
defined as follows.\vspace{3mm}

{\scriptsize\sffamily
\quad{\bf Input: } text $T={}$BANANA\eos
\vspace{2mm}

\tabcolsep=.25em
\quad\begin{minipage}[t]{26mm}
\raggedright
%\centering
  1. Build a matrix with the text \emph{rotations} as rows
\begin{center}
\sffamily
    \begin{tabular}{ccccccc}
 %   \multicolumn{1}{c}{$F$} &&&&&\multicolumn{1}{c}{}& \multicolumn{1}{c}{}\\
    B&A&N&A&N&A&\eos\\
    A&N&A&N&A&\eos&B\\
    N&A&N&A&\eos&B&A\\
    A&N&A&\eos&B&A&N\\
    N&A&\eos&B&A&N&A\\
    A&\eos&B&A&N&A&N\\
    \eos&B&A&N&A&N&A\\
  \end{tabular}
  \end{center}
\end{minipage}
% \parbox[t]{6mm}{\centering \mbox{}\\\rule{0pt}{20mm}
% $\overset{\textrm{sort}}{\Rightarrow}$
% }%
\hfill
\begin{minipage}[t]{28mm}
\centering 2. Sort the rows
\begin{center}
\sffamily
    \begin{tabular}{|c|ccccc|c|}
    \multicolumn{1}{c}{$F$} &&&&&\multicolumn{1}{c}{}& \multicolumn{1}{c}{$L$}\\
    \cline{1-1}\cline{7-7}
    \eos&B&A&N&A&N&A\\
    A&\eos&B&A&N&A&N\\
    A&N&A&\eos&B&A&N\\
    A&N&A&N&A&\eos&B\\
    B&A&N&A&N&A&\eos\\
    N&A&\eos&B&A&N&A\\
    N&A&N&A&\eos&B&A\\
    \cline{1-1}\cline{7-7}
  \end{tabular}
\end{center}
\end{minipage}
\vspace{2mm}

\quad{\bf Output: } BWT $L={}$ANNB\eos AA (the last column)
\vspace{3mm}
}

\footnotesize The properties of the BWT make it easier to
compress than the original text. It is used as the first stage in many
compression programs including the widely used bzip2 (thus the b).

\end{textblock} 

% Second column
\begin{textblock}{60}(65,20)
  {\sffamily\normalsize{\color{sciorange}INVERSE BWT}}\vspace{1mm}\\
  \footnotesize 
  Define
  $
    \rank(j)=\big|\{i \mid i<j \textrm{ and }
    L[i]=L[j]\}\big|.
  $

The BWT can be inverted as follows.
\vspace{3mm}

\scriptsize\sffamily
\quad{\bf Input: } BWT $L={}$ANNB\eos AA
\vspace{2mm}

\quad\begin{minipage}{55mm}
\scriptsize\sffamily
1. Compute $C$ and \rank\ arrays by scanning $L$
\scriptsize\sffamily
\begin{center}
\hspace*{-5mm}\input{permutation}  
\end{center}
2. Starting at $L[i]={}$\eos, follow the permutation:
\[
i \mapsto C[L[i]]+\rank(i)
\]
\hspace{2.3mm}Output $L[i]$ at each step
\end{minipage}
\vspace{2mm}

\scriptsize\sffamily
\quad{\bf Output: } reverse text $T^R={}$\eos ANANAB

\end{textblock}
